\documentclass[12pt]{article}

% Packages
\usepackage{amsmath}
\usepackage{hyperref}
\usepackage{graphicx}
\usepackage[margin=1in]{geometry}

\title{Beam and Ball Controller Design - Speaker Notes}
\author{Alexander Brown}
\date{December 2, 2024}

\begin{document}

\section*{Slide 1: Project Overview}
\textbf{Detailed Notes:}
\begin{itemize}
    \item \textbf{Problem Statement:} The ball-and-beam system is a fundamental example of an unstable system in control theory:
    \begin{itemize}
        \item Small disturbances can cause significant deviations in the ball's position.
        \item A feedback controller is necessary to stabilize the system and return the ball to its desired position.
    \end{itemize}
    \item \textbf{Objective:}
    \begin{itemize}
        \item Design a feedback controller to adjust the beam angle and stabilize the ball at the center.
        \item Use a 3D simulation model (Simscape Multibody) to analyze the system’s real-world behavior.
    \end{itemize}
    \item \textbf{Technical Approach:}
    \begin{itemize}
        \item Use \textbf{Newtonian mechanics} to derive force-based equations of motion.
        \item Use \textbf{Lagrangian mechanics} to represent energy-based system dynamics.
        \item Linearize the equations around the equilibrium point (\(x = 0, \theta = 0\)) for control design.
        \item Implement a Linear Quadratic Regulator (LQR) to optimize control performance.
    \end{itemize}
\end{itemize}

---
\newpage
\section*{Slide 2: Challenges and Control Goals}
\textbf{Detailed Notes:}
\begin{itemize}
    \item \textbf{System Challenges:}
    \begin{itemize}
        \item The system is inherently unstable. Small angular changes (\(\theta\)) amplify into large displacements (\(x\)).
        \item Dynamics of the ball and beam are coupled, meaning the motion of one affects the other.
        \item The nonlinear nature of the dynamics complicates control design.
    \end{itemize}
    \item \textbf{Control Goals:}
    \begin{itemize}
        \item Stabilize the ball at the center position (\(x = 0\)).
        \item Minimize transient response metrics:
        \begin{itemize}
            \item \textbf{Overshoot:} Limit the extent to which the ball overshoots its target position.
            \item \textbf{Settling Time:} Reduce the time required for the ball to reach and remain near the target.
            \item \textbf{Steady-State Error:} Ensure the ball remains as close to the center as possible over time.
        \end{itemize}
    \end{itemize}
    \item The diagram shows how forces interact within the system, highlighting the instability and complexity of control.
\end{itemize}

\newpage
\section*{Slide 3: Degrees of Freedom and Modeling Approach}
\textbf{Detailed Notes:}
\begin{itemize}
    \item \textbf{Degrees of Freedom:}
    \begin{itemize}
        \item \textbf{Translational motion (\(x\)):} Describes the ball’s position along the beam axis.
        \item \textbf{Rotational motion (\(\theta\)):} Refers to the beam’s angle relative to the horizontal.
    \end{itemize}
    \item \textbf{Modeling Approach:}
    \begin{itemize}
        \item \textbf{Newtonian Mechanics:}
        \begin{itemize}
            \item \textbf{\(F = ma\):} Governs the translational motion of the ball, where \(F\) is the net force, \(m\) is the ball’s mass, and \(a\) is the acceleration.
            \item \textbf{\(\tau = I \alpha\):} Describes the beam’s rotational motion, where \(\tau\) is the torque, \(I\) is the moment of inertia, and \(\alpha\) is the angular acceleration.
        \end{itemize}
        \item \textbf{Lagrangian Mechanics:}
        \begin{itemize}
            \item The system dynamics are formulated based on energy:
            \begin{itemize}
                \item \textbf{Kinetic Energy (\(T\)):} Represents energy due to motion of the ball and the beam.
                \item \textbf{Potential Energy (\(V\)):} Accounts for gravitational effects acting on the ball.
            \end{itemize}
            \item The Lagrangian is defined as:
            \[
            \mathcal{L} = T - V
            \]
            \item Equations of motion are derived using:
            \[
            \frac{d}{dt}\left(\frac{\partial \mathcal{L}}{\partial \dot{q}}\right) - \frac{\partial \mathcal{L}}{\partial q} = 0
            \]
            where \(q\) represents generalized coordinates (e.g., \(x\) and \(\theta\)).
        \end{itemize}
    \end{itemize}
\end{itemize}

---
\newpage
\section*{Slide 4: System Parameters and State-Space Representation}
\textbf{Detailed Notes:}
\begin{itemize}
    \item \textbf{System Parameters:}
    \begin{itemize}
        \item \textbf{Beam:} 
        \begin{itemize}
            \item Length (\(L = 1.0 \, \mathrm{m}\)).
            \item Width (\(w = 0.05 \, \mathrm{m}\)).
            \item Height (\(h = 0.1 \, \mathrm{m}\)).
        \end{itemize}
        \item \textbf{Ball:}
        \begin{itemize}
            \item Mass (\(m = 0.5 \, \mathrm{kg}\)).
            \item Radius (\(r = 0.05 \, \mathrm{m}\)).
            \item Moment of inertia (\(I = 0.02 \, \mathrm{kg{\cdot}m^2}\)).
        \end{itemize}
        \item \textbf{Gravity:} \(g = 9.81 \, \mathrm{m/s^2}\), which influences the potential energy and torque acting on the system.
    \end{itemize}
    \item \textbf{State-Space Representation:}
    \begin{itemize}
        \item The linearized equations of motion are expressed in state-space form:
        \[
        \dot{x} = A x + B u, \quad y = C x + D u
        \]
        \item Matrices \(A\) and \(B\) capture the system’s dynamics:
        \[
        A = \begin{bmatrix} 
        0 & 1 & 0 & 0 \\
        0 & 0 & \frac{g}{L} & 0 \\
        0 & 0 & 0 & 1 \\
        0 & 0 & -\frac{m g r}{I} & 0 
        \end{bmatrix}, \quad 
        B = \begin{bmatrix} 
        0 \\ 
        0 \\ 
        0 \\ 
        \frac{1}{I} 
        \end{bmatrix}.
        \]
        \item Linearization is necessary to simplify control design while ensuring accuracy for small deviations.
    \end{itemize}
\end{itemize}

---
\newpage
\section*{Slide 5: Controller Design}
\textbf{Detailed Notes:}
\begin{itemize}
    \item \textbf{Objective:} Design a controller to balance system performance and control effort by minimizing the cost function:
    \[
    J = \int_0^\infty \left( x^T Q x + u^T R u \right) dt
    \]
    \begin{itemize}
        \item The first term (\(x^T Q x\)) penalizes state deviations (e.g., ball position and beam angle).
        \item The second term (\(u^T R u\)) penalizes excessive control input to ensure energy efficiency and practicality.
    \end{itemize}
    \item \textbf{Design Parameters:}
    \begin{itemize}
        \item \(Q = \text{diag}([200, 10, 10, 10])\):
        \begin{itemize}
            \item Emphasizes the importance of precise ball position (\(x\)).
            \item Penalizes deviations in beam angle (\(\theta\)).
        \end{itemize}
        \item \(R = 1\): Ensures smooth and efficient control inputs.
    \end{itemize}
    \item \textbf{MATLAB Implementation:}
    \begin{itemize}
        \item Feedback gain (\(K\)) is computed using MATLAB's LQR function:
        \[
        K = \texttt{lqr}(A, B, Q, R)
        \]
        \item The control law is defined as:
        \[
        u = -Kx
        \]
        where \(u\) is the control input, and \(x\) represents the system states.
        \item This approach ensures optimal trade-offs between control precision and energy usage.
    \end{itemize}
\end{itemize}

---
\newpage
\section*{Slide 6: MATLAB Integration}
\textbf{Detailed Notes:}
\begin{itemize}
    \item MATLAB was used for the numerical computation, design of the controller, and performance evaluation.
    \item \textbf{Purpose:}
    \begin{itemize}
        \item Define system parameters, including the ball and beam properties.
        \item Derive and implement the state-space representation of the system.
        \item Design the controller using the \texttt{lqr} function to calculate feedback gains (\(K\)).
    \end{itemize}
    \item \textbf{MATLAB Features:}
    \begin{itemize}
        \item Performance metrics:
        \begin{itemize}
            \item \textbf{IAE (Integral Absolute Error):} Measures the cumulative error magnitude to ensure precise control.
            \item \textbf{ISE (Integral Square Error):} Emphasizes larger errors to penalize instability.
            \item \textbf{ITAE (Integral Time-weighted Absolute Error):} Focuses on reducing sustained errors, improving settling time.
        \end{itemize}
        \item Assess the transient response:
        \begin{itemize}
            \item Overshoot quantifies how much the ball exceeds its desired position.
            \item Settling time measures how quickly the system stabilizes.
        \end{itemize}
    \end{itemize}
    \item \textbf{Outputs:}
    \begin{itemize}
        \item Plots for:
        \begin{itemize}
            \item Ball position (\(x\)) and beam angle (\(\theta\)).
            \item Control input (\(u\)) over time to evaluate controller performance.
        \end{itemize}
        \item Metrics provide a quantitative way to compare the performance of different controllers.
    \end{itemize}
\end{itemize}

---
\newpage
\section*{Slide 7: Simulation Results}
\textbf{Detailed Notes:}
\begin{itemize}
    \item \textbf{Objective:} Evaluate the performance of the LQR controller using simulations.
    \item \textbf{State Trajectories:}
    \begin{itemize}
        \item The ball’s position stabilizes within 5 seconds after minor disturbances.
        \item The beam angle (\(\theta\)) adjusts dynamically to counteract disturbances and stabilize the ball.
        \item Both plots demonstrate the controller’s ability to achieve stability with minimal overshoot.
    \end{itemize}
    \item \textbf{Control Input:}
    \begin{itemize}
        \item Smooth control input (\(u\)) indicates the controller does not demand excessive effort, ensuring practical implementation.
        \item Oscillation-free behavior validates the LQR controller's effectiveness over alternatives like pole placement.
    \end{itemize}
    \item \textbf{Performance Metrics:}
    \begin{itemize}
        \item \textbf{IAE (0.06657):} Indicates minimal cumulative error, ensuring precise stabilization.
        \item \textbf{ISE (0.00494):} Reflects small deviations and penalizes larger errors effectively.
        \item \textbf{ITAE (0.02813):} Demonstrates efficient settling and minimal sustained errors.
    \end{itemize}
    \item Visual representations, including trajectories and control input, illustrate the system’s dynamic response under LQR control.
\end{itemize}

---

\section*{Slide 8: Simulink System Diagram}
\textbf{Detailed Notes:}
\begin{itemize}
    \item \textbf{Purpose:} Simulate the real-world behavior of the ball-and-beam system using Simulink and the Simscape Multibody toolbox.
    \item \textbf{System Components:}
    \begin{itemize}
        \item \textbf{Global Configuration:} Includes world frame and solver configuration for numerical simulations.
        \item \textbf{Ball and Beam Dynamics:}
        \begin{itemize}
            \item Models the interaction between the ball’s motion and the beam’s rotation.
            \item Captures nonlinear dynamics realistically.
        \end{itemize}
        \item \textbf{Integral Compensation:} Corrects for steady-state errors by integrating position errors over time.
        \item \textbf{LQR Controller:} Provides feedback gains (\(K\)) for stabilizing the system.
        \item \textbf{States Subsystem:} Tracks key variables, including \(x, \dot{x}, \theta, \dot{\theta}\).
    \end{itemize}
    \item \textbf{Benefits:}
    \begin{itemize}
        \item Realistic visualization of the ball and beam motion.
        \item Validates MATLAB-based controller design in a 3D environment.
    \end{itemize}
    \item \textbf{Outputs:} Includes dynamic animations and plots that align with experimental results, enabling intuitive analysis.
\end{itemize}

---
\newpage
\section*{Slide 9: Challenges and Future Work}
\textbf{Detailed Notes:}
\begin{itemize}
    \item \textbf{Challenges:}
    \begin{itemize}
        \item Parameter sensitivity: Variations in ball mass or beam length can destabilize the system.
        \item Linearization limitations: The model’s accuracy decreases for large beam angles (\(\theta\)).
        \item Pole placement was ineffective:
        \begin{itemize}
            \item Led to oscillations and high control effort.
            \item LQR provided a more balanced approach.
        \end{itemize}
    \end{itemize}
    \item \textbf{Future Work:}
    \begin{itemize}
        \item Incorporate external disturbances (e.g., friction) for robust testing.
        \item Use Kalman filtering for noise reduction in state estimation.
        \item Extend the control system for dynamic reference tracking.
        \item Investigate nonlinear control techniques to improve large-angle stability.
    \end{itemize}
\end{itemize}

---

\end{document}
